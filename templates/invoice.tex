\documentclass[a4paper, article, oneside, hidelinks, 10pt]{memoir}

% Encoding and fonts
\usepackage[utf8]{inputenc}
\usepackage[T1]{fontenc}
\usepackage{sfmath}
\usepackage{amsmath}
\usepackage[final]{microtype}
\usepackage[sfdefault]{ClearSans}

% Language
\usepackage[ngerman]{babel}

% Layout
\usepackage[a4paper, left=2.5cm, right=2.5cm, top=2cm, bottom=3cm]{geometry}
\usepackage{fancyhdr}

% Colors and hyperlinks
\usepackage{xcolor}
\usepackage{hyperref}

% Graphics
\usepackage{graphicx}

% SI units
\usepackage{siunitx}

% Table-related packages
\usepackage{tabularray}

% QR code generation
\usepackage[yyyymmdd]{datetime}
\usepackage[qrscheme=epc,replace-umlauts=true,qrsize=20mm]{qrbill}

% Euro symbol
\usepackage{eurosym}
\DeclareSIUnit{\EUR}{\text{\euro}}

% SI units setup
\sisetup{
  locale = DE,
  round-mode = places,
  round-precision = 2,
  group-minimum-digits=3,
  detect-all = true,
}
\UseTblrLibrary{siunitx}

% QR code setup
\SetupQrBill{
  QRType=BCD,
  Version=002,
  CodingType=1,
  Trailer=SCT,
}

% Miscellanous styling
\pagestyle{empty}


% Set up the header and footer
\pagestyle{fancy}
\fancyhf{}

% Header settings
\renewcommand{\headrulewidth}{0pt}

% Footer settings
\renewcommand{\footrulewidth}{1pt}
\lfoot{\footnotesize
\begin{tblr}{width=\textwidth, colspec={X[l]X[l]X[l]}, leftsep=0pt, rightsep=0pt}
	{(((company.name))) \\ (((address.street))) \\ (((address.zip))) (((address.city)))} & {(((company.phone))) \\ \href{mailto:(((company.email)))}{(((company.email)))} \\ \href{https://(((company.website)))}{(((company.website)))}} & {Finanzamt: (((company.tax_office))) \\ Steuernummer: (((company.tax_number)))} \\
\end{tblr}
}

\setlength{\parindent}{0cm}

\begin{document}

% Issuer
\begin{raggedleft}
	\small
	\textbf{(((company.name)))} \\
	(((address.street))) \\
	(((address.zip))) (((address.city))) \\
	(((company.phone))) \\
	\href{mailto:(((company.email)))}{(((company.email)))} \\
	\href{https://(((company.website)))}{(((company.website)))} \\
\end{raggedleft}

\vspace{1cm}

% Recipient and invoice information
\begin{minipage}[t]{0.6\textwidth}
	{\scriptsize \underline{(((company.name))), (((address.street))), (((address.zip))) (((address.city)))}}

	\medskip

	{\large \textbf{\dots}} \\
	\dots \\
	\dots \\
	\dots
\end{minipage}
\begin{minipage}[t]{0.4\textwidth}
	\small
	\begin{tabbing}
		\hspace{0.2\textwidth} \= \hspace{0.2\textwidth} \kill
		Rechnungsnummer: \` \hfill (((invoice.number))) \\
		Kundennummer: \` \hfill \dots \\
		Datum: \` \hfill \dots \\
		Zahlungsfrist: \` \hfill \dots \\
	\end{tabbing}
\end{minipage}

\vspace{2cm}

\chapter*{Rechnung (((invoice.number)))}

Sehr geehrte Damen und Herren,

\medskip

meine Leistungen stelle ich Ihnen wie folgt in Rechnung.

% Items
\begin{longtblr}[entry = none, label = none, note{*} = {Umsatzsteuerfreie Leistungen gemäß §19 UStG.}]{width=\textwidth, colspec={cXrr*{2}{Q[si={table-format=4.2},r]}}, vlines, hlines, row{1}={guard,font=\bfseries}, row{Z}={guard,font=\bfseries,gray9}, rowhead=1, rowfoot=1, abovesep=4pt, belowsep=4pt}
	Pos.                                               & Bezeichnung   & Menge                                                                            & Einheit & Einzel \texteuro & Gesamt \texteuro \\
	((* for item in items *))
	                                                   & (((item[0]))) & ((* if item[2] is not none and item[2]|length != 0 *))(((item[2])))((* endif *)) &         &                  & 0                \\
	((* endfor *))
	test                                               &               &                                                                                  &         & 0                & 0                \\
	\SetCell[c=5]{l} \textbf{Gesamtbetrag}\TblrNote{*} &               &                                                                                  &         &                  & \num{10}         \\
\end{longtblr}

Bitte überweisen Sie den Betrag von \dots bis zum \dots an die folgende Bankverbindung. \textit{Der dargestellte QR-Code kann zur automatischen Übernahme der Daten in Ihr Online-Banking genutzt werden.}

\bigskip

% Bank information
\begin{minipage}[c]{0.7\textwidth}
	\small
	\begin{tblr}{width=\textwidth, colspec={X[l]X[l]}, leftsep=0pt, rightsep=0pt}
		Kontoinhaber:     & (((company.name)))   \\
		IBAN:             & (((bank.iban)))      \\
		Bank:             & (((bank.bank_name))) \\
		Verwendungszweck: & (((invoice.number))) \\
	\end{tblr}
\end{minipage}
\begin{minipage}[c]{0.3\textwidth}
	\hfill
	\QRbill*[
		Account=(((bank.iban | replace(' ', '')))),
		Name=(((company.name))),
		Amount=EUR10.00,
		Message={(((invoice.number)))},
	]
\end{minipage}

\bigskip

Vielen Dank für die gute Zusammenarbeit.

\end{document}
